\documentclass{ata-calico}
\usepackage{indentfirst}
\pagenumbering{arabic}
\begin{document}

\maketitle

\pauta{Discussão}

\section{Estado de Greve x Greve}
A assembleia começa deixando claro a diferença entre estado de greve e greve. Dá-se exemplos de cursos que já estão em greve, como a Arq, e cursos que estão em estado de greve, ou seja, esperando uma decisão oficial dos estudantes para entrar em greve (talvez a partir do da 10). Explica-se que com a decisão dos estudantes os professores não são obrigados a parar,  e os alunos da ARQ, junto com o DCE, estão colocando pressão para que os professores parem de dar aula.

\section{Por que agora?}
Robson pergunta qual a vantagem de entrarmos em greve agora. Leonardo responde que a aula vai parar de qualquer jeito, e que a greve é uma forma de ocupar o nosso espaço, para que mantenhamos a universidade funcionando, não deixando os cortes acontecerem. Cauê diz que o cenário atual é preocupante, que anteriormente o RU ia funcionar apenas para isentos, a universidade ia esvaziar e não teríamos chance de construir uma luta. Luis diz que mesmo em greve, a aula deve continuar, mas fora do plano de ensino, para usarmos o tempo que estaríamos em aula para construir uma defesa da universidade, levando para fora da universidade porque ela é tão importante. 

\subsection{Visão da sociedade numa greve agora}
Balde diz que a universidade já vai parar de qualquer forma e deveríamos mostrar como queremos estudar, fazendo greve só depois acabar. A imagem de uma greve agora pode ser péssimo em no ponto de vista da sociedade. É questionado os prós e contras de uma greve.

Seis comenta que a visão da sociedade não mudaria agora, já que mesmo com os alunos estudando, a visão da sociedade ainda é a mesma.

Julien comenta como a greve é um instrumento de luta e que é necessária uma greve ativa. Comenta também que a visão pejorativa dos estudantes vai continuar, da filosofia ao direito. A greve é pela luta contra os cortes, para negar tudo o que eles vêm colocando.

Robson reitera sobre a visão pejorativa. Cauê e Paloma comentam que não é possível esperar, pois a pessoas não conseguirão se manter aqui. 

Arthur Capaverde diz que temos que usar a greve como \textbf{oportunidade de falar com a sociedade}, já que as pessoas vão pensar coisas ruins de qualquer forma. 

Trombeta fala sobre as bolsas que foram canceladas, enfatizando que as pessoas já não conseguem estudar, mesmo querendo. 

Tiz diz que o nosso objetivo é ter uma educação de qualidade, garantindo esta para futuras gerações. Comenta sobre os médicos do HU que fizeram um trabalho de conscientização no hospital, explicando o que a universidade faz, explicando que é isso que devemos fazer.

\section{Como fazer a greve?}
Bruno questiona quão bem nós vamos fazer essa greve.

Ravi diz que devemos aderir a greve, e fazer como fizeram no HU, um trabalho de base com a sociedade. 

Leonardo diz que a questão dos professores deveria ser um encaminhamento. Diz que ele próprio é um exemplo que não conseguiria se manter sem a bolsa estudantil e a permanência. A greve permitiria várias pessoas que atualmente não conseguem participar ativamente, por aulas ou outros motivos, participar das mobilizações.

Cauê diz que a arquitetura, assim que declarou greve, veio em massa fazer cartazes e faixas. Hoje também a Odontologia declarou greve e foi participar da reunião do DCE. Ou seja, as pessoas estão tendo tempo e se mobilizando. Comenta também que além dos cortes esse ano, o orçamento do ano que vem já foi reduzido em 30\% de verba, e que UFSC irá cortar das bolsas e permanência. A Greve deve ser feita contra o ensino precarizado.

Luis diz que iremos amanhã ecidir o que fazer, em assembleia estudantil, na greve e fora dela. Fala que mais de 50 cursos já deliberaram por greve ou estado de greve. Comenta sobre o UFSC na praça, que mostra pras pessoas o que acontece dentro da universidade em praça pública. Luis reforça que todos os presentes a favor devem participar da greve, já que foi tirado em coletivo o que a maioria quer. 

Leonardo pergunta se nós iremos nos juntar a outros cursos e que nós como estudantes estamos falando para nós mesmos como a universidade importante, que deveríamos bolar estratégias para mudar isso. 

Capa diz que precisamos saber o que fazer depois da greve, e quais ferramentas nós temos. 

\section{Pautas partidárias}
Gustavo diz que é a favor da greve mas é contra as bandeiras levantadas pelos partidos. Diz que é preciso pressionar os professores, vindo do Calico, mas que não se pode cancelar as aulas.

Balde diz que deveríamos colocar como pauta exclusivamente "contra os cortes na UFSC", para evitar palanques políticos, dizendo que pessoas defendendo outras pautas pode afastar outras pessoas. 

Arthur diz que o movimento deve ser feito de forma clara, sem camisetas de outras pautas. Gabi diz que devemos ter uma identidade em comum.  

Seis diz que não devemos ver as bandeiras partidárias como inimigos e sim como pessoas tentando ajudar.

\section{Extensão e como mostrar o que queremos/fazemos?}
Bruno diz que devemos mostrar pros professores qual o nosso objetivo com a greve e diz que é muito difícil explicar para a população o que nós fazemos.


Robson diz que os professores também podem falar sobre seus laboratórios, e que não se deve privar o conhecimento fazendo com que as aulas não aconteçam, por que isso seria contra nós mesmos. 

É comentado sobre a a administração da universidade, que pode também ser mal administrada, deveríamos nos perguntar por que chegou a esse ponto. Comenta-se que as pessoas que não conseguirem se manter aqui vão trabalhar ou voltar para suas cidades, e não vir pra UFSC se mobilizar. 

Trombeta diz que não vê como é possível negociar com o governo, e que apenas queremos estudar com dignidade. Fala novamente sobre as bolsas de CNPQ, e que mesmo se liberarem as verbas para esse ano, há muitas coisas em jogo. Diz que podemos também ter ideias para a greve e não apenas seguir o que os outros dizem.

Leonardo se desculpa pela fala sobre o Bolsonaro. Acredita ser impossível convencer os eleitores ferrenhos a mudar de ideia. Com relação a como manter as pessoas aqui: a greve seria o primeiro passo. Fazer uma ocupação pode ser a solução. Outro ponto: projetos de extensão. Cita o ESUS como exemplo. Devemos fazer aulas como extensão, não dentro do calendário acadêmico.

Seis diz também que temos grana pra fazer greve agora, ou seja, as pessoas ainda estariam aqui, e que devemos fazer isto. Diz também que se alguém não consegue se mobilizar, essa greve não é só para uma pessoa, e sim para o coletivo e diz que deveríamos falar sim sobre nosso curso e pesquisas para a população. 

Coquinho diz que esse momento é preciso construir com todos, que devemos aprender outras coisas além do que aprendemos no nosso curso, e que precisamos da participação de todos.

Cauê diz que a participação dos partidos não se tornam a pauta do movimento, apenas ajudam a mobilizar. Comenta também sobre as despesas que a universidade não pode mexer, dizendo que o que foi bloqueado foi o custeio, que afeta outros setores. Pede para que construamos as entidades de base e fala que é necessário também explicar sobre as pesquisas de base, como as de física e matemática, que nós produzimos.

Tiz diz que temos que sintetizar as pesquisas de forma a conseguir explicar pra sociedade. Comenta também sobre a possibilidade de ocupação. Fala sobre montar um calendário de greve. 

Julien comenta dos 95 funcionários terceirizados demitidos.

É falado mais uma vez como a greve é importante para as pessoas poderem participar. 

Isabella diz que independente da nossa decisão, a greve vai acontecer. Estamos decidindo se vamos nos juntar e ajudar a construir ela, pois não estamos lutando pelo individual e sim pelo coletivo. 

Leonardo diz que devemos pensar no futuro da universidade, que há pessoas que virão e irão depender da desta. 
\pauta{Encaminhamentos}
\begin{itemize}
\item \textbf{Exigir um posicionamento dos professores:} Aprovado por unanimidade
\item \textbf{Exigir uma assembleia dos cursos do CTC:} Aprovado por maioria
\item \textbf{Fazer um grupo de mobilização da computação:} É concordado em fortificar o Calico ao invés de criar uma comissão externa.
\item \textbf{Calico organizar uma reunião até o fim da semana:} Aprovado por unanimidade
\item \textbf{Aderir a greve, independente de hoje ou amanha:} Aprovado por maioria
\item \textbf{Aderir a greve hoje:} 22 votos
\item \textbf{Aderir a greve dependendo de amanhã:} 39 votos 
\item \textbf{Divulgação das noticias estudantis: Calico deve se organizar para divulgar melhor as noticias:} aprovada por unanimidade.
\item \textbf{Deixar para discutir ocupação depois:} Unanimidade
\item \textbf{Calendário de greve:} Criar na reunião dessa semana do Calico, junto com o plano de ação, a discussão sobre ocupação e outros pontos
\end{itemize}

\presentes {em anexo}
\end{document}
